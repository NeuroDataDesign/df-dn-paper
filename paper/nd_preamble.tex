% \usepackage{draftwatermark}
% \SetWatermarkText{Technical Report}
% \SetWatermarkScale{0.4}

\usepackage{comment}
\usepackage{lipsum}
\usepackage[margin=1in, footskip=36pt]{geometry}

%%% For author, thanks, affil
%\usepackage[noblocks]{authblk}
\usepackage{endnotes}
\renewcommand{\notesname}{Affiliations}

\usepackage{titlesec}
\usepackage{titletoc}
\usepackage{pgfplotstable}
%%=================
%% Define Headings and section task
%%=================
\titleclass{\task}{straight}[\section]
\newcounter{task}
\renewcommand{\thetask}{\arabic{task}}
\titleformat{\task}[hang]
    {\normalfont\LARGE\bfseries}{Task \thetask:}{1em}{}


\titleformat*{\task}{\color{header1}\bfseries}

\titlecontents{task}
              [3.8em] % ie, 1.5em (chapter) + 2.3em
              {}
              {\contentslabel{2.3em}}
              {\hspace*{-2.3em}}
              {\titlerule*[1pc]{.}\contentspage}


\titlespacing*{\section}{0ex}{1ex}{1ex}
\titlespacing*{\subsection}{0ex}{1ex}{1ex}
\titlespacing*{\subsubsection}{0ex}{1ex}{1ex}
\titlespacing*{\paragraph}{0ex}{1ex}{1ex}
\titlespacing*{\subparagraph}{0pt}{1ex}{1ex}
\titlespacing*{\task}{0em}{1ex}{1ex}


%%% for footnote affiliation block
\newcommand\blfootnote[1]{%
  \begingroup
  \renewcommand\thefootnote{}\footnote{#1}%
  \addtocounter{footnote}{-1}%
  \endgroup
}

%%% Bib and citations
\usepackage[sort&compress,comma,square,numbers]{natbib}

%%% mathfonts, equation environments and eqref, ...
\usepackage{amsfonts,amsmath,amssymb}
\usepackage{bbm}

%%% For enumitem with special global options
\usepackage{multicol}
\usepackage{enumitem}
\setlist[enumerate]{wide, labelindent=1cm,  noitemsep}
% \setlist[enumerate]{noitemsep}
\setlist[itemize]{noitemsep}
\setlist[description]{noitemsep}

%%% For colors and sout and hhlines
\usepackage{hhline}


%%%%%%%%%%% EDITING MACROS %%%%%%%%%%%%%
% \usepackage{todonotes}
\RequirePackage{ulem}
\newcommand{\add}[1]{\textcolor{blue}{[#1]}}
\newcommand{\rep}[2]{\textcolor{brown}{[\sout{#1} $|$ #2]}}
\newcommand{\dele}[1]{\textcolor{red}{\sout{#1}}}

%% Set up Tables
%%%%%%%%%%%%%%%%%%%%%%%%%%%%%%%%%%%%%===%
\definecolor{tableheadcolor}{gray}{0.92}
% Following is taken from Werner: http://tex.stackexchange.com/a/33761/3061
% and modified for my needs
%
% Command \topline consists of a (slightly modified)
% \toprule followed by a \heavyrule rule of colour tableheadcolor
% (hence, 2 separate rules)
\newcommand{\topline}{ %
        \arrayrulecolor{blue1}\specialrule{0.1em}{\abovetopsep}{0pt}%
        \arrayrulecolor{tableheadcolor}\specialrule{\belowrulesep}{0pt}{0pt}%
        \arrayrulecolor{blue1}}
% Command \midline consists of 3 rules (top colour tableheadcolor, middle colour black, bottom colour white)
\newcommand{\midtopline}{ %
        \arrayrulecolor{tableheadcolor}\specialrule{\aboverulesep}{0pt}{0pt}%
        \arrayrulecolor{blue1}\specialrule{\lightrulewidth}{0pt}{0pt}%
        \arrayrulecolor{white}\specialrule{\belowrulesep}{0pt}{0pt}%
        \arrayrulecolor{blue1}}
% Command \bottomline consists of 2 rules (top colour
\newcommand{\bottomline}{ %
        \arrayrulecolor{white}\specialrule{\aboverulesep}{0pt}{0pt}%
        \arrayrulecolor{blue1} %
        \specialrule{\heavyrulewidth}{0pt}{\belowbottomsep}}%


\newcommand{\midheader}[2]{%
        \midrule\topmidheader{#1}{#2}}
\newcommand\topmidheader[2]{\multicolumn{#1}{c}{\textsc{#2}}\\%
                \addlinespace[0.5ex]}

\pgfplotstableset{normal/.style ={%
        header=true,
        string type,
        font=\small,
    	column type={p{.092\textwidth}},
        every head row/.style={
            before row={\topline\rowcolor{tableheadcolor}},
            after row={\midtopline}
        },
        every odd row/.style={
            before row={\rowcolor{blue1}}
        },
        every last row/.style={
            after row=\bottomline
        },
        col sep=&,
        row sep=\midrule
    }
}

\usepackage{multirow}


%%% for \toprule
\usepackage{booktabs}
\usepackage{titlesec}
\usepackage{fancyhdr}
\usepackage{fullpage}

% Define LaTeX friendly Fonts
%==============
\RequirePackage{titlesec}
\RequirePackage[font={footnotesize},{singlelinecheck=false}]{caption}
\captionsetup{belowskip=-5pt}

\usepackage[T1]{fontenc}
\usepackage[scaled]{helvet}
\usepackage[scaled=1.10, med]{zlmtt}
\renewcommand\familydefault{\sfdefault}


%%% For begin{algorithm}
\usepackage{algorithm}
\usepackage{algpseudocode}

\newcommand{\Linefor}[2]{%
    \State \algorithmicfor\ {#1}\ \algorithmicdo\ {#2} \algorithmicend\ \algorithmicfor%
}
\newcommand{\Lineif}[2]{%
    \State \algorithmicif\ {#1}\ \algorithmicdo\ {#2} \algorithmicend\ \algorithmicif%
}



%%%%%%%%%%%%%%%%%%%%%%%%%%%%%%%%%%%%%%%%%
%%%%%%%%%%% MACROS %%%%%%%%%%%%%%%%%%%%%%
%%%%%%%%%%%%%%%%%%%%%%%%%%%%%%%%%%%%%%%%%


%%%%%%%%%%% ALGORITHM NAMES %%%%%%%%%%%%%
\providecommand{\sct}[1]{{\sc \texttt{#1}}}

\newcommand{\Idt}{\sct{Idt}}
\newcommand{\Svd}{\sct{Svd}}
\newcommand{\Pca}{\sct{Pca}}
\newcommand{\Fld}{\sct{Fld}}
\newcommand{\Lda}{\sct{Lda}}
\newcommand{\eig}{\sct{eig}}
\newcommand{\Lol}{\sct{Lol}}
\newcommand{\Lal}{\sct{Lal}}
\newcommand{\Qoq}{\sct{Qoq}}
\newcommand{\Lrl}{\sct{Lrl}}
\newcommand{\Lfl}{\sct{Lfl}}
\newcommand{\Faq}{\sct{Faq}}
\newcommand{\qr}{\sct{qr}}

\newcommand{\Mgc}{\sct{Mgc}}
\newcommand{\Mgcp}{\Mgc$_P$}
\newcommand{\Mgcd}{\Mgc$_D$}
\newcommand{\Mgcm}{\Mgc$_M$}
\newcommand{\Hhg}{\sct{Hhg}}
\newcommand{\Dcorr}{\sct{Dcorr}}
\newcommand{\Dcov}{\sct{Dcov}}
\newcommand{\Mcorr}{\sct{Mcorr}}
\newcommand{\Mantel}{\sct{Mantel}}
\newcommand{\Mic}{\sct{Mic}}
\newcommand{\Hsic}{\sct{Hsic}}
\newcommand{\Pearson}{\sct{Pearson}}
\newcommand{\Kendall}{\sct{Kendall}}
\newcommand{\Spearman}{\sct{Spearman}}
\newcommand{\RV}{\sct{RV}}
\newcommand{\CCA}{\sct{Cca}}
\newcommand{\Sporf}{\sct{Sporf}}
\newcommand{\Mf}{\sct{Mf}}
\newcommand{\MF}{\sct{RF}}
\newcommand{\ICC}{\sct{ICC}}



%%%%%%% MATH OPERATORS %%%%%%%%%%%%

\providecommand{\ve}[1]{\boldsymbol{#1}}
\providecommand{\ma}[1]{\boldsymbol{#1}}
\providecommand{\deter}[1]{\lvert #1 \rvert}
\newcommand{\argmax}{\operatornamewithlimits{argmax}}
\newcommand{\argmin}{\operatornamewithlimits{argmin}}
\newcommand{\argsup}{\operatornamewithlimits{argsup}}
\newcommand{\arginf}{\operatornamewithlimits{arginf}}
\newcommand{\T}{^{\ensuremath{\mathsf{T}}}} % transpose
\providecommand{\norm}[1]{\ensuremath{\left \lVert#1 \right  \rVert}}
\providecommand{\abs}[1]{\ensuremath{\left \lvert #1 \right \rvert}}
% \providecommand{\mat}[1]{\left[ #1 \right]}
% \newcommand{\trans}[1]{{#1}^{\ensuremath{\mathsf{T}}}}    % \newcommand{\transpose}[1]{{#1}^{\ensuremath{\mathsf{T}}}}
\newcommand{\from}{{\ensuremath{\colon}}}           % :
\newcommand{\trace}[1]{{\ensuremath{\operatorname{tr}\!\left(#1\right)}}}           % :
%
\providecommand{\mc}[1]{\mathcal{#1}}
\providecommand{\tmc}[1]{\tilde{\mathcal{#1}}}
\providecommand{\mb}[1]{\boldsymbol{#1}}
\providecommand{\ms}[1]{\mathsf{#1}}
\providecommand{\mt}[1]{\widetilde{#1}}
\providecommand{\mbb}[1]{\mathbb{#1}}
\providecommand{\mv}[1]{\vec{#1}}
\providecommand{\mh}[1]{\hat{#1}}
\providecommand{\wh}[1]{\widehat{#1}}
\providecommand{\mhv}[1]{\mh{\mv{#1}}}
\providecommand{\mvh}[1]{\mv{\mh{#1}}}
\providecommand{\mhc}[1]{\hat{\mathcal{#1}}}
\providecommand{\mbc}[1]{\mb{\mathcal{#1}}}
\providecommand{\mvc}[1]{\mv{\mathcal{#1}}}
\providecommand{\mtc}[1]{\widetilde{\mathcal{#1}}}
\providecommand{\mth}[1]{\mt{\mh{#1}}}
\providecommand{\mht}[1]{\mh{\mt{#1}}}
\providecommand{\mhb}[1]{\hat{\boldsymbol{#1}}}
\providecommand{\whb}[1]{\widehat{\boldsymbol{#1}}}
\providecommand{\mvb}[1]{\vec{\boldsymbol{#1}}}
\providecommand{\mtb}[1]{\widetilde{\boldsymbol{#1}}}
\providecommand{\mbt}[1]{\widetilde{\boldsymbol{#1}}}
\providecommand{\mvc}[1]{\vec{\mathcal{#1}}}
% \newcommand{\D}[2]{\frac{\partial #1}{\partial #2}}
\newcommand{\dd}[2]{\frac{\partial ^2 #1}{\partial #2 ^2}}
\newcommand{\DDD}[3]{\frac{\partial ^2 #1}{\partial #2 \partial #3}}
\newcommand{\Di}[2]{\frac{\partial ^i #1}{\partial #2 ^i}}



%%%%%%%%% SHORT HAND %%%%%%%%%%
\newcommand{\website}{\url{https://neurodata.io/}}
\newcommand{\jhu}{Johns Hopkins University}
\newcommand{\jvemail}{\href{mailto:jovo@jhu.edu}{jovo@jhu.edu}}
\newcommand{\jv}{Joshua T.~Vogelstein}

\newcommand{\Vr}{V_{reset}}
\newcommand{\Vl}{V_{leat}}
\newcommand{\eqdef}{\overset{\triangle}{=}}
\newcommand{\grad}{\nabla}
\newcommand{\Hess}{\nabla\nabla}
\newcommand{\defn}{\overset{\triangle}{=}}

\newcommand{\rto}{\leftarrow}
\newcommand{\iid}{\overset{iid}{\sim}}
\newcommand{\knn}{$k$NN}

\newcommand{\elegans}{\emph{C. elegans} }

\newcommand{\Lik}{\mathcal{L}}
\newcommand{\Cae}{[\widehat{\text{Ca}}^{2+}]}
\newcommand{\Cav}{\ve{C}}%[\ve{\text{Ca}}^{2+}]}
\newcommand{\sml}{\sqrt{\ma{\lambda}}}
\newcommand{\ml}{\ma{\lambda}}
\newcommand{\nw}{\widehat{n}}
\newcommand{\nv}{\vec{n}}
\newcommand{\Ae}{\widehat{A}}
\newcommand{\te}{\widehat{\tau}}
\newcommand{\maxn}{\max_{\ve{n}: n_t \geq 0}}
% \newcommand{\V}{\text{Var}}

\newcommand{\PmcP}{P \in \mc{P}}
\newcommand{\mP}{\mathbb{P}}

% \newcommand{\dvs}{\dot{\bs}_t}
% \newcommand{\dvw}{\dot{\bw}_t}
% \newcommand{\dvx}{\dot{\bx}_t}
% \newcommand{\dvy}{\dot{\by}_t}

\newcommand{\ft}{f_{\ve{\thet}}}
\newcommand{\gt}{g_{\ve{\thet}}}
\newcommand{\hht}{h_{\thetn}}

\newcommand{\Real}{\mathbb{R}}
\newcommand{\Ind}{\mathbb{I}}

\newcommand{\wconv}{\overset{i.p.}{\rightarrow}}
\newcommand{\sconv}{\overset{i.p.}{\rightarrow}}
\newcommand{\conv}{\rightarrow}
\newcommand{\pconv}{\overset{p}{\conv}}
\newcommand{\mcE}{\mathcal{E}}
\newcommand{\mcT}{\mathcal{T}}
\newcommand{\mcG}{\mathcal{G}}
\newcommand{\mcM}{\mathcal{M}}
\newcommand{\mcL}{\mathcal{L}}
\newcommand{\hatmcE}{\widehat{\mcE}}
\newcommand{\hatp}{\widehat{p}}
\newcommand{\hatP}{\widehat{P}}
\newcommand{\hatQ}{\widehat{Q}}
\newcommand{\hatL}{\widehat{L}}
\newcommand{\mhP}{\widehat{\PP}}
\newcommand{\tildeA}{\widetilde{A}}
\newcommand{\defeq}{\overset{\triangle}{=}}


\DeclareMathOperator{\Pmat}{\mathbf{P}}
\DeclareMathOperator{\veta}{\mathbf{\mb{v}}}
\DeclareMathOperator*{\minimize}{\mathrm{minimize}}
\DeclareMathOperator*{\maximize}{\mathrm{maximize}}
% \DeclareMathOperator*{\mb{v}mod}{\mathbf{\mb{v}}}


%%%%%%% LATIN LETTERS


\newcommand{\bA}{\mb{A}}
\newcommand{\bB}{\mb{B}}
\newcommand{\bD}{\mb{D}}
\newcommand{\bE}{\mb{E}}
\newcommand{\bI}{\mb{I}}
\newcommand{\bP}{\mb{P}}
\newcommand{\bS}{\mb{S}}
\newcommand{\bU}{\mb{U}}
\newcommand{\bV}{\mb{V}}
\newcommand{\bW}{\mb{W}}
\newcommand{\bX}{\mb{X}}
\newcommand{\bY}{\mb{Y}}
\newcommand{\bZ}{\mb{Z}}

\newcommand{\ba}{\mb{a}}
\renewcommand{\ba}{\mb{b}}
\newcommand{\bd}{\mb{d}}
\newcommand{\be}{\mb{e}}
\newcommand{\bp}{\mb{p}}
\newcommand{\bs}{\mb{s}}
\newcommand{\bu}{\mb{u}}
\newcommand{\bv}{\mb{v}}
\newcommand{\bw}{\mb{w}}
\newcommand{\bx}{\mb{x}}
\newcommand{\by}{\mb{y}}
\newcommand{\bz}{\mb{z}}


\newcommand{\Aa}{\mathbb{A}}
\newcommand{\BB}{\mathbb{B}}
\newcommand{\CC}{\mathbb{C}}
\newcommand{\DD}{\mathbb{D}}
\newcommand{\EE}{\mathbb{E}}           % expected value
\newcommand{\FF}{\mathbb{F}}
\newcommand{\GG}{c}
\newcommand{\HH}{\mathbb{H}}
\newcommand{\II}{\mathbb{I}}           % indicator function
\newcommand{\LL}{\mathbb{L}}
\newcommand{\MM}{\mathbb{M}}
\newcommand{\NN}{\mathbb{N}}
\newcommand{\PP}{\mathbb{P}}
\newcommand{\QQ}{\mathbb{Q}}
\newcommand{\RR}{\mathbb{R}}
\newcommand{\SSS}{\mathbb{S}}
\newcommand{\VV}{\mathbb{V}}
\newcommand{\WW}{\mathbb{W}}
\newcommand{\XX}{\mathbb{X}}
\newcommand{\YY}{\mathbb{Y}}
\newcommand{\ZZ}{\mathbb{Z}}




\newcommand{\Qs}{Q}
\newcommand{\mcS}{\mc{S}}
\newcommand{\mcU}{\mc{U}}

\newcommand{\mbd}{\ensuremath{\mb{d}}}
\newcommand{\mbD}{\ensuremath{\mb{D}}}
\newcommand{\mbx}{\ensuremath{X}}
\newcommand{\mbX}{\ensuremath{\mb{X}}}
\newcommand{\mby}{\ensuremath{Y}}
\newcommand{\mbY}{\ensuremath{\mb{Y}}}

\newcommand{\mtbd}{\mtb{d}}
\newcommand{\mtbD}{\mtb{D}}
\newcommand{\mtbx}{\mtb{x}}
\newcommand{\mtbX}{\mtb{X}}
\newcommand{\mtby}{\mtb{y}}
\newcommand{\mtbY}{\mtb{Y}}



\DeclareMathOperator{\Ri}{\mathbf{\R}^{-1}}
\DeclareMathOperator{\A}{A}
\DeclareMathOperator{\W}{\mathbf{W}}
\DeclareMathOperator{\V}{\mathbf{V}}
%\DeclareMathOperator{\U}{\mathbf{U}}
%\DeclareMathOperator{\C}{\mathbf{C}}
\DeclareMathOperator{\uvec}{\mathbf{u}}
\DeclareMathOperator{\D}{\mathbf{D}}
\DeclareMathOperator{\Q}{\mathbf{Q}}
\DeclareMathOperator{\R}{R} %\mathbf{P}}
\DeclareMathOperator{\Y}{\mathbf{Y}}
%\DeclareMathOperator{\BB}{\mathbf{B}}
\DeclareMathOperator{\Hmat}{\mathbf{H}}
\DeclareMathOperator{\Gmat}{\mathbf{G}}
\DeclareMathOperator{\X}{\mathbf{X}}
\DeclareMathOperator{\Cmat}{C} %\mathbf{L}}\providecommand{\ms}[1]{\mathsf{#1}}

\DeclareMathOperator*{\Ymod}{\mathbf{\Y}}
\DeclareMathOperator*{\Bmod}{\mathbf{B}}
\DeclareMathOperator*{\Hmod}{\mathbf{H}}
\DeclareMathOperator*{\Lmod}{\mathbf{L}}
\DeclareMathOperator*{\Xmod}{\mathbf{\X}}


%%% THETA %%%%

\newcommand{\bth}{\ve{\theta}}
\newcommand{\hth}{\mh{\theta}}
\newcommand{\htth}{\mh{\theta}}
\newcommand{\bhth}{\mh{\ve{\theta}}}
\newcommand{\thetn}{\ve{\theta}}
\newcommand{\thet}{\thetn}
\newcommand{\theth}{\widehat{\ve{\theta}}}
\newcommand{\theto}{\ve{\theta}'}
\newcommand{\wht}{\widehat{\thet}}
\newcommand{\wtt}{\widetilde{\thet}}
\newcommand{\vth}{\ve{\thet}}
\newcommand{\vTh}{\ve{\Theta}}
\newcommand{\hvth}{\widehat{\ve{\thet}}}
\newcommand{\bTh}{\ve{\Theta}}
\newcommand{\hbth}{\widehat{\thet}}
\newcommand{\tbth}{\tilde{\bth}}



% \newcommand{\p}{P_{\bth}}
\newcommand{\pold}{P_{\bth'}}
\newcommand{\pk}{P_{\widehat{\ve{\theta}}^{(k)}}}
\newcommand{\pT}{P_{\thetn_{Tr}}} %\thetn_T
\newcommand{\pO}{P_{\thetn_o}} %\thetn_o
% \newcommand{\Q}{Q(\thetn,\theto)}
% \newcommand{\m}{m^{\ast}}
% \newcommand{\q}{q(\ve{H}_t)}
\newcommand{\Ca}{[\text{Ca}^{2+}]}


%%%%%% GREEK LETTERS

\newcommand{\del}{\delta}
\newcommand{\sig}{\sigma}
\newcommand{\lam}{\lambda}
\newcommand{\gam}{\gamma}
\newcommand{\eps}{\varepsilon}

\newcommand{\Del}{\Delta}
\newcommand{\Sig}{\Sigma}
\newcommand{\Lam}{\Lambda}
%\newcommand{\Gam}{\Gamma}

\newcommand{\bSig}{\ve{\Sigma}}
\newcommand{\bOm}{\ve{\Omega}}
\newcommand{\bLam}{\ve{\Lambda}}
\newcommand{\bPhi}{\ve{\Phi}}
\newcommand{\bPsi}{\ve{\Psi}}

\newcommand{\bmu}{\ve{\mu}}
\newcommand{\bal}{\ve{\alpha}}
\newcommand{\bpi}{\ve{\pi}}
\newcommand{\bkap}{\ve{\kappa}}
\newcommand{\bdel}{\ve{\delta}}
\newcommand{\bphi}{\ve{\phi}}
\newcommand{\bpsi}{\ve{\psi}}



\DeclareMathOperator{\Delti}{\mathbf{\Delta}^{-1}}
\DeclareMathOperator{\Delt}{Q} %\mathbf{\Delta}}
\DeclareMathOperator{\Gam}{\mathbf{\Gamma}}
\DeclareMathOperator{\Gami}{\mathbf{\Gamma}^{-1}}
\DeclareMathOperator{\Sigb}{\mathbf{\Sigma}}


\newtheorem{Def}{Definition}
\newcommand\floor[1]{\lfloor#1\rfloor}